% !TeX root = ../main.tex
% Add the above to each chapter to make compiling the PDF easier in some editors.

\chapter{Streamline Integration}\label{chapter:streamlines}
The approach to the generation of streamlines, plays an important role in achieving good results of the detected similarity between the reference and network representations.
Different seeding strategies as well as integration methods, each come with their own advantages and disadvantages.

We chose to focus on different approaches for the creation and comparison of streamlines, as opposed to pathlines or streaklines.
This means each sample taken during the generation of a streamline is taken from the same timestep. 
% add math p_n+1 = p_n + f(t, p_n)
In contrast for a streakline each velocity sampled for the streamline would be on a timestep later than the previous.
% add math p_n+1 = p_n + f(t+h, p_n)
A comparison of streamlines, streaklines and pathlines can be seen in X.
% streamline vs streakline vs pathline
% what is used in the following (eval vs network)
\section{General Methodology}
The generation of streamlines is generally divided into two steps: Seeding and Integration
During seeding, the initial points in space R3 and time R1 for the streamlines are generated.
The seeding method affects coverage of the velocity field. An even coverage can be achieved using a uniform random/deterministic approach. 
Alternatively an approach using importance sampling can assist with highlighting more salient regions of the field.

The chosen integration method affects the accuracy of the streamlines according to the chosen step size.
More advanced integration methods like rk45 will net a higher accuracy using the same step size in exchange for higher computational requirements.
The used methods are highlighted in more detail in the following sections.
Seed initial points:
Generate next set of points
repeat
result is Tensor of Shape x
\section{Seeding Strategy}
The choice of seeding strategy can have a profound effect on streamline coverage and legibility.
Choosing a good seeding strategy to cover large portions of the field, as well as in particular highlighting singularities and regions is an area of active research. Examples include.
Since the purpose of our SRN is the visual representation of the original field, these considerations fit with our goal.
The following methods focus on spatial distribution, as opposed to distribution among timesteps. 
The latter will be discussed in (later section/further research)
\subsection{Uniform Random}
The most naive approach to seeding is to use a uniform random distribution.
%math where x ~ 0-1 for R^3
By definition this approach has a very even distribution of seed points. An example of it can be seen in x.
It does however lack any effect towards highlighting more salient regions especially during visualization.
As such, a sparse field with regions of low interest might generate a significant number of streamlines with low movement.
\subsection{Uniform Deterministic}
During visualization the number of seed points has to be kept relatively low to keep visibility high. 
As such the actual distribution of streamlines when using a uniform random method might be less uniform than desired.
A deterministic, evenly spaced distribution can be used in this case.
%math 
An example of this distribution can be seen in.
In general this approach is mostly useful for testing. 
In comparisons (especially if used during training) a random distribution is more desirable in order to reduce overfitting.
\subsection{Importance Sampling using local Entropy}
Our final approach to seeding is generating a local entropy field at each time step, which can then be applied in importance sampling.
An entropy field can help highlight more salient regions in the field and increase the number of seed points generated in these areas.

The generation of the entropy field is a preprocessing step which has to be applied to each time step of the reference that we want to generate seeds on.
\subsubsection{Generation of the Entropy field}
How is Entropy calculated here?:
  Sphere of equal regions
  neighborhood size
  Use vector length or not?
Rejection Sampling
  intensity
  min probability
uneven coverage, focus on "important" regions
\subsection{Comparison}
On Speed and Accuracy

\section{Integration Methods}

\subsection{Euler-Integration}
procedure
no additional samples needed
\subsection{Runge-Kutta}
some additional samples
error is O()
\subsection{Runge-Kutta-Fehlberg}
adaptive stepsize
more accurate
requires higher number of samples
individual?
similarity might suffer under adaptive stepsize
\chapter{Streamline Similarity}
\section{Prior Considerations}
Masking, Time-relation (streamline vs pathline), adaptive stepsize
\section{L2-Distance}
Procedure, Limitations (stepsize, length, mask)
decent since comparing streamlines with same start location
\section{Bag-Of-Features}
instead of distance generate a set of features for each point on the streamline
compare these using clustered representative feature vectors
\subsection{Choice of Features}

\subsection{Spatially Sensitive Bag-of-Features}
\section{Comparison}
On Speed and Accuracy
(Look up Error estimation)